\chapter{Corne 键盘配置分析 (Corne Keyboard Config Analysis)}

\section{概述 (Overview)}
本文详细分析了 \lstinline|corne.keymap| 和 \lstinline|corne_adhoc_v11.keymap| 两个配置文件的按键设置。这两个配置均基于 ZMK 固件,为 Corne 这一 42 键(或 36 键)分体键盘设计。分析重点在于层级结构 (Layer Structure)、宏定义 (Macro Definitions) 以及标准版与 Ad-hoc 版本之间的核心差异。

\section{通用宏定义 (Common Macro Definitions)}
两个配置文件共享了一套强大的宏定义,主要针对 macOS 生态系统进行了深度优化:

\paragraph{系统功能宏}
\begin{itemize}
    \item \lstinline|macos_screenshot| (\lstinline|mac_sc|): 触发 macOS 区域截图 (Cmd + Ctrl + Shift + 4)。
    \item \lstinline|macos_screenshot_with_annotation| (\lstinline|mac_sca|): 触发带注释的 macOS 区域截图 (Cmd + Shift + 4)。
    \item \lstinline|macos_input| (\lstinline|mac_in|): 切换输入法 (Ctrl + Alt + Space)。
\end{itemize}

\paragraph{编辑与导航宏}
\begin{itemize}
    \item \lstinline|alt_backspace|: 删除一个词 (Alt + Backspace)。
    \item \lstinline|ctrl_alt_left/right/up/down|: 带有组合键的导航,常用于窗口管理。
    \item \lstinline|cmd_1| 至 \lstinline|cmd_5|: 快速切换应用窗口或标签页。
\end{itemize}

\section{标准版配置分析 (Standard Config Analysis)}
在 \lstinline|corne.keymap| 中,配置相对简洁,采用了典型的三层结构:

\paragraph{默认层 (DEFAULT/DFT)}
基于 QWERTY 布局。使用了大量的 Mod-Tap 功能,例如 \lstinline|TAB| 键长按为 \lstinline|LGUI|,\lstinline|BSPC| 长按为 \lstinline|RGUI|,\lstinline|ESC| 长按为 \lstinline|LCTRL|。这种设计在小键盘上极大提高了修饰键的可访问性。

\paragraph{低层 (Lower Layer/LWR)}
主要负责数字输入 (1-0) 和方向键导航。此外,还将 \lstinline|mac_sc| 和 \lstinline|mac_sca| 放置在该层,方便快速截图。

\paragraph{高层 (Raise Layer/RSE)}
集中了所有的符号键 (Symbols),以及 RGB 灯效控制和蓝牙 (Bluetooth) 设备切换。

\section{Ad-hoc v11 版配置分析 (Ad-hoc v11 Config Analysis)}
\lstinline|corne_adhoc_v11.keymap| 展示了更复杂的演进,它被称为 ``Neovim Layer'' 配置,显然是为重度代码开发优化的:

\paragraph{层级扩张}
该版本定义了 6 个层:\lstinline|DFT| (0), \lstinline|LWR| (1), \lstinline|RSE| (2), \lstinline|TMP| (3), \lstinline|FUC| (4), \lstinline|OPR| (5)。

\paragraph{功能分离}
与标准版不同,v11 将功能控制 (RGB, BT) 独立到了 \lstinline|FUC| (Function) 层。默认层通过 \lstinline|lt FUC RET| 将回车键在长按时映射为该功能层。

\paragraph{操作层 (Operation Layer/OPR)}
这是一个独特的创新。通过 \lstinline|to OPR| 切换,该层 is 默认层的副本,但在拇指键上增加了 \lstinline|LGUI|。这可能用于在特定任务(如频繁使用 GUI 快捷键)中临时改变键盘行为,而无需一直长按修饰键。

\section{核心差异对比 (Core Differences)}
\begin{itemize}
    \item \textbf{修饰键策略}: \lstinline|corne.keymap| 右侧拇指使用 \lstinline|RGUI|,而 v11 倾向于使用 \lstinline|RCTRL| 和 \lstinline|RALT|。
    \item \textbf{层管理}: v11 引入了 \lstinline|to| 动作(切换层)而非仅仅是 \lstinline|mo| 或 \lstinline|lt|(临时开启层),提供了更持久的状态切换能力。
    \item \textbf{专业化}: v11 的命名 (Neovim Layer) 和 OPR 层的设计,体现了从通用办公向高效专业编程(尤其是 Neovim 用户)的转变。
\end{itemize}
