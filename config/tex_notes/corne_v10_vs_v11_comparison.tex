\chapter{Corne 键盘配置对比分析: v10 vs v11 (Corne Keymap Comparison: v10 vs v11)}

\section{概述 (Overview)}
本文对比分析了 \lstinline|corne_prod_v10.keymap|(生产版 v10)与 \lstinline|corne_adhoc_v11.keymap|(Ad-hoc v11)的差异。这两个版本代表了从标准通用配置向高度专业化(特别是针对 Neovim 优化)配置的演进。

\section{生产版 v10 分析 (Analysis of Production v10)}
\lstinline|corne_prod_v10.keymap| 与基础的 \lstinline|corne.keymap| 几乎完全一致,采用的是成熟且稳定的三层架构:
\begin{itemize}
    \item \textbf{层级结构}: 包含 \lstinline|DFT| (0), \lstinline|LWR| (1), \lstinline|RSE| (2)。
    \item \textbf{修饰键设计}: 使用 \lstinline|mt LGUI TAB| 和 \lstinline|mt RGUI BSPC| 作为左右两侧的拇指修饰键。
    \item \textbf{功能布局}: 数字层 (\lstinline|LWR|) 包含导航,符号层 (\lstinline|RSE|) 包含 RGB 和蓝牙控制。这种布局逻辑清晰,适合大多数通用办公场景。
\end{itemize}

\section{Ad-hoc v11 分析与对比 (Comparison with Ad-hoc v11)}
\lstinline|corne_adhoc_v11.keymap| 在 v10 的基础上进行了大幅度的重构,其核心差异体现在以下几个方面:

\paragraph{层级结构的急剧扩张}
v11 将层级数量从 3 个增加到了 6 个。新增了 \lstinline|FUC| (Function)、\lstinline|OPR| (Operation) 和 \lstinline|TMP| (Template) 层。这种细分允许用户将 ``系统控制''(如蓝牙、灯效)与 ``输入逻辑''(如数字、符号)完全解耦。

\paragraph{拇指键逻辑的重定义}
在 v11 中,拇指键的逻辑变得极为复杂且强大:
\begin{itemize}
    \item \textbf{回车键}: 在 v10 中,回车是一个单纯的按键。在 v11 中,它变成了 \lstinline|lt FUC RET|,长按直接进入功能控制层。
    \item \textbf{修饰键切换}: v11 将右侧默认的 \lstinline|RGUI| 替换为 \lstinline|RCTRL|,这更符合专业程序员(尤其是 Linux/Unix 环境)的操作习惯。
\end{itemize}

\paragraph{操作层 (OPR) 的引入}
这是 v11 最显著的特征。通过 \lstinline|to OPR| 切换,该层是默认层的副本,但在拇指键上增加了 \lstinline|LGUI|。这可能用于在特定任务(如频繁使用 GUI 快捷键)中临时改变键盘行为,而无需一直长按修饰键。

\section{结论 (Conclusion)}
\begin{itemize}
    \item \textbf{v10 (Production)}: 侧重于稳定性和通用性,是经过验证的最佳实践布局。
    \item \textbf{v11 (Ad-hoc)}: 侧重于效率和深度定制。通过增加层级深度 and 引入状态切换 (Layer Toggling),它极大地扩展了 42 键键盘的操作空间,特别适合需要频繁切换不同快捷键模式的开发者。
\end{itemize}
